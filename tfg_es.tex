
%----------------------------------------------------------------------------------------
%	PACKAGES AND OTHER DOCUMENT CONFIGURATIONS
%----------------------------------------------------------------------------------------

\documentclass[fleqn,12pt]{UICArticle} % Document font size and equations flushed left
\usepackage[spanish]{babel} % Specify a different language here - english by default
\usepackage{listings}
\usepackage{dirtytalk}
\usepackage{setspace}

%---------------------------------------------------------------------------------------
%   Default font family and spacing
%----------------------------------------------------------------------------------------
\onehalfspacing
\renewcommand{\familydefault}{\sfdefault}

%----------------------------------------------------------------------------------------
%	WORK INFORMATION
%----------------------------------------------------------------------------------------

\University{Universitat Internacional de Catalunya}
\TypeOfWork{Trabajo fin de grado}
\Degree{Grado en Medicina y Cirugía}
\Date{16 de Mayo de 2017}
\PaperTitle{El uso de conectomas basados en particiones estructuro-funcionales del cerebro para la clasificación automatizada de subtipos de esclerosis múltiple.} % Article title
\Authors{Gabriel de Maeztu\textsuperscript{1}*} % Authors
\affiliation{\textsuperscript{1}\textit{Facultad de Medicina, Universitat Internacional de Catalunya, Barcelona, Spain}} 
\affiliation{*\textbf{Datos de contacto}: gabriel.mp@uic.es} % Corresponding author

\Keywords{
Magnetic Resonance Imaging [D008279] ---
Brain Mapping [D001931] ---
Connectomics [D063132] ---
Multiple Sclerosis [D009103] ---
Humans [D006801]
} % Keywords - if you don't want any simply remove all the text between the curly brackets
\newcommand{\keywordname}{Palabras clave} % Defines the keywords heading name
\newcommand{\abstractnameEng}{Abstract – English} % Defines the heading of the second abstract

%----------------------------------------------------------------------------------------
%	ABSTRACT
%----------------------------------------------------------------------------------------
\Abstract{ 
La necesidad de desarrollar nuevos marcadores para la esclerosis multiple (EM) ha empujado a la búsqueda de métodos alternativos para medir la afectación de la enfermedad sobre la materia blanca (MB). La conectómica, que estudia la conectividad cerebral como si de una red o grafo se tratara, está aportando nuevos marcadores, demostrando una correlación entre la afectación de la red y la afectación de la EM. Actualmente los conectomas están demostrando su posible potencial en la clasificación temprana de los pacientes a los diferentes fenotipos EM o paciente sano. El objetivo de este estudio es comparar el método de clasificación de pacientes con EM propuesto por Kocevar et al. con una variente donde las particiones cerebrales utilizadas para la creación del conectoma están basadas en la parcelación estructuro-funcional del cerebro. Para ello se reutilizarán los datos de Kocevar et al. (67 pacientes y 26 controles) y se compararán la sensibilidad y especificidad entre estos dos métodos de clasificación de pacientes.
 
~}

\AbstractEng{
The need to develop new markers for multiple sclerosis (MS) has led to the search for alternative methods to measure disease involvement on white matter (WM). Connectivism, which studies brain connectivity as if it were a network or graph, is contributing new markers, demonstrating a correlation between the network's involvement and the involvement of MS. Currently connectomas are demonstrating their potential in the early classification of patients to different MS or healthy patient phenotypes. The objective of this study is to compare the method of classification of MS patients proposed by Kocevar et al. With a variant where the brain partitions used for the creation of the connectam are based on the structural-functional splitting of the brain. For this purpose, data from Kocevar et al. (67 patients and 26 controls) and will compare the sensitivity and specificity between these two methods of classifying patients.
~}
% TODO: Mejorar abstract
% quitar toda la parte de como se generan y añadir la parte de rehacer comparando con otros datos

%----------------------------------------------------------------------------------------

\begin{document}

\begingroup
\singlespacing
% Makes all text pages the same height
\flushbottom 

% Print the title and abstract box
\maketitle
\thispagestyle{empty} 
\clearpage

% Blank page
\thispagestyle{empty} 
\null\newpage

% Abstract page
\makeabstract
\thispagestyle{empty} 
\clearpage

% Print the contents section
\tableofcontents
\thispagestyle{empty} 
\clearpage
\endgroup

%----------------------------------------------------------------------------------------
%	ARTICLE CONTENTS
%----------------------------------------------------------------------------------------
\begingroup

%%%%%
\setlength{\parindent}{1em}
\setlength{\parskip}{0.8em}
%%%%%

\section{Introducción}

\subsection{Problema}

La imagen por resonancia magnética nuclear (IRMN) se ha convertido en la herramienta referente en el diagnóstico de la esclerosis multiple (EM). Su alta sensibilidad permite obtener un pronóstico de la evolución de la enfermedad \cite{Filippi1994} y ser un predictor de las recurrencias clínicas y de la gravedad de la discapacidad futura \cite{Filippi1995}.
 
El manejo clínico de la EM depende de la evolución de la sintomatología así como de diferentes marcadores obtenidos a partir de las IRMN o de escalas clínicas como la EDSS y la MSFC. Para poder mejorar el manejo de estos pacientes, existe la necesidad de desarrollar marcadores adicionales y alternativos. La IRMN cerebral tiene el potencial para desarrollar nuevos marcadores que permitan monitorear la enfermedad y personalizar el tratamiento, individualizándolo a la evolución de cada paciente \cite{Rio2017}.

\subsection{Marco teorico}

\subsubsection{La esclerosis múltiple y su diagnóstico}

La EM es una enfermedad neurodegenerativa crónica autoinmune y es la causa más común de discapacidad en el adulto joven \cite{Polman2011}. Afecta a la mielina de las neuronas del sistema nervioso central produciendo una desmielinización e inflamación que son consideradas la base de la patología \cite{Br2005}.

Existen criterios no invasivos para diagnosticar la enfermedad, como los establecidos por McDonald et al.\cite{Lublin2014} pero la prueba de referencia con mayor fiabilidad sigue siendo la biopsia o la autopsia. Estudios histopatológicos han demostrado la existencia de daño en la materia blanca (MB) durante las primeras etapas de la EM, antes de que se puedan distinguir lesiones hiperintensas en las secuencias de $T_2$ de la IRMN \cite{Beer2016, Moll2011, Miller2003}.

La pobre correlación entre las lesiones visibles en la IRMN y el grado de afectación de los pacientes es un problema conocido como la “paradoja clínico-radiológica de la EM” \cite{Barkhof2002}. Esta paradoja es la que ha motivado la búsqueda de métodos complementarios para la detección y clasificación de pacientes con EM. 

\subsubsection{Fenotipos de la Esclerosis múltiple}

Actualmente, la EM se clasifica en los siguientes fenotipos en base a su evolución : clínicamente aislada (CIS), remitente-recurrente (RR), progresiva primaria (PP) y secundariamente progresiva (SP); según la clasificación revisada de Lublin et al \cite{Lublin2014}. La determinación de estos fenotipos se lleva a cabo a través de diferentes escalas e IRMN según los criterios revisados de McDonald \cite{Polman20112}. La neurodegeneración secundaria de las lesiones provocadas por la EM es a día de hoy difícil de cuantificar por IRMN y por ello difícil de correlacionar con los fenotipos clínicamente descritos \cite{Miller2003}. 

\begin{figure}[h]
	\centering
	\includegraphics[width=10cm]{img/mri}
	\caption{Representación visual de un corte de IRMN}
	\label{fig:voxeles}
\end{figure}

\subsubsection{Procesamiento de las IRMN}

El análisis sistematizado de la IRMN abre la puerta a nuevas herramientas diagnósticas y pronósticas. El procesamiento matemático de la IRMN ha permitido detectar y cuantificar cambios aparentemente no visibles para los radiólogos y correlacionarlos directamente con el daño histopatológico producido por la EM \cite{Beer2016, Nedjati-Gilani2017}. Esto es posible gracias a que la IRMN, captura vóxeles \{fig.\ref{fig:voxeles}\}, unidad cúbica que compone un objeto tridimensional. Estos vóxeles son parte de una matriz, que puede ser analizada de manera sistemática.
 
En base a diferentes análisis que se pueden realizar sobre esta misma matriz es posible estudiar la conectividad cerebral. Gracias a ello están surgiendo nuevas formas de medir alteraciones en las redes neuronales \cite{Johansen2006}.

Una de estas herramientas que permite estudiar la conectividad de la MB cerebral son los conectomas. Se tratan de recreaciones tridimensionales de los tractos neuronales, calculadas a partir de la IRMN generada con la técnica de resonancia magnética de difusión (DWI). En la \{fig.\ref{fig:connectome}\} se puede apreciar una de estas recreaciones.

\begin{figure}[h]
	\centering
	\includegraphics[width=9cm]{img/connectome}
	\caption{Reconstrucción 3D de los tractos de la MB (arriba) y la selección de algunos de los tractos (abajo).}
	\label{fig:connectome}
\end{figure}


\subsubsection{Procesamiento de los conectomas}

Los conectomas se obtienen haciendo uso de algoritmos de tractografía. Estos algoritmos tienen en cuenta las direcciones de difusión del agua en la imagen de DWI para crear una reconstrucción virtual de los tractos neuronales \cite{Mori2002}. Esto es posible gracias a la medición de la magnitud y orientación de la difusión de las moléculas de agua dentro de los tejidos cerebrales. La orientación y magnitud de la difusión de las moléculas de agua está supeditada a la morfología de los axones \cite{Johansen2006} y por ello es posible determinar la morfología aproximada de los tractos neuronales.

Las mediciones que se pueden realizar en los conectomas son indirectas pero su naturaleza no invasiva y su facilidad de medición han abierto un nuevo campo de estudio. Uno de las diferentes análisis que se pueden llevar a cabo en los conectomas es tratar el conectoma como una red o grafo que describe la conectividad del cerebro \cite{Fornito, Bullmore2009}.

Si se asigna una etiqueta al origen y destino de cada uno de los tractos que componen el conectoma es posible estudiarlo como si de una red se tratara. Para ello es necesario asignar las etiquetas, dividiendo el cerebro  en diferente áreas o parcelaciones y asignando una etiqueta a cada una de ellas. Clásicamente se han utilizado las parcelaciones basadas en la anatómia del cráneo o las parcelaciones basadas en la citoarquitectura de las neuronas (áreas de Brodmann) \cite{Destrieux2010}.

\begin{figure}[h]
	\centering
	\includegraphics[width=7cm]{img/brain_graph}
	\caption{Etiquetado de las diferentes particiones formando una red en el cerebro. \textit{Representación de I. Diez Palacio}}
	\label{fig:brein_graph}
\end{figure}

\subsubsection{Análisis de grafos (redes)}

Las redes (llamadas a menudo grafos en teoría de la computación), han demostrado un gran potencial para estudiar sistemas biológicos complejos, incluyendo el sistema nervioso, y están comenzando a revelar principios fundamentales de la arquitectura y funcionalidad del cerebro. El lenguaje matemático que describe y cuantifica estas redes se llama teoría de grafos \cite{Mori2002}. 
 
Los grafos constan de varios elementos, que aplicados a este caso pueden resumirse del siguiente modo  \{fig.\ref{fig:graph}\}:
\begin{enumerate}[noitemsep]
\item \textbf{Vértices (nodos):} Representan las parcelaciones/regiones cerebrales de interés.
\item \textbf{Enlaces (aristas):} Representan los tractos/conexiones entre las regiones del cerebro.
\end{enumerate}

\begin{figure}[h]
	\centering
	\includegraphics[width=10cm]{img/graph}
	\vspace{5mm} 
	\caption{Representación de un grafo simple}
	\label{fig:graph}
\end{figure}

El procesamiento de los conectomas permite obtener un grafo del cerebro \{fig.\ref{fig:network}\}, en el que se pueden realizar mediciones objetivas y cuantificables de las redes neuronales \cite{Mori2002}. Las etiquetas asignada al inicio y final de cada uno de los tractos del conectoma, son las que se utilizan para determinar el nodo al que pertenece cada conexión en el grafo. De este modo, hay tantos nodos como parcelaciones en el cerebro y tantos enlaces como conexiones neuronales (en valor relativo), obteniendo así la red que será estudiada.

Existen diferentes propiedades matemáticas que permiten describir y cuantificar un grafo. Estas medidas se pueden utilizar para describir las siguientes propiedades del cerebro \cite{Rubinov2010}:
\begin{enumerate}[noitemsep]
\item \textbf{Segregación:} La posibilidad de que  un procesamiento ocurra dentro de grupos densamente interconectados de regiones cerebrales.
\item \textbf{Integración:} La capacidad de combinar rápidamente la información de regiones del cerebro separadas.
\item \textbf{Centralidad:} La importancia de ciertas regiones en la red.
\item \textbf{Resistencia:} Mide la vulnerabilidad de la red ante perturbaciones.
\item \textbf{Motivo:} Mide la frecuencia de aparición de ciertos patrones.
\end{enumerate}

\begin{figure}[p]
	\centering
	\includegraphics[width=\linewidth]{img/network}
	\vspace{5mm} 
	\caption{Grafo basado en las particiones de Destrieux et al.\cite{Destrieux2010}}
	\label{fig:network}
\end{figure}

Estos datos sirven para describir una red neuronal, y por ello se pueden utilizar para discriminar diferentes individuos en base a una patología \cite{Muthuraman2016}. Para realizar esta diferenciación, es necesario utilizar herramientas estadísticas que puedan buscar las diferencias entre estos descriptores y en base a ellas realizar una clasificación del individuo a una categoria.


\subsubsection{Aprendizaje automatizado (Machine Learning)}

El aprendizaje automatizado es la capacidad de un algoritmo de aprender de la experiencia acumulada al realizar una tarea para después estimar el resultado de nuevos casos \cite{Friedman1997}.

La experiencia se considera que son los datos aportados al algoritmo, en este caso serían por un lado los parametros que describen el grafo y  por el otro el diagnóstico establecido según los criterios de McDonald. La tarea es dar un fenotipo a partir del análisis de los parametros del grafo. Tras introducir unos datos iniciales, es decir, el grafo y el fenotipo de muchos pacientes, el algoritmo será capaz de dar un diagnóstico a partir del grafo de un nuevo paciente en base la experiencia previamente adquirida.

\subsubsection{Recapitulación}

Los pacientes con EM necesitan de nuevos marcadores para el diagnóstico y monitorización de su patología. La IRMN es una herramienta ampliamente adoptada en el manejo de la EM y nuevas técnicas de análisis sobre estas imágenes como la tractografía han abierto la puerta al análisis de la conectividad cerebral. Al verse la conectividad afecta debido a la fisiopatología de la EM es posible clasificar a los pacientes haciendo uso de nuevas herramientas estadísticas como el aprendizaje automatizado en base a los descriptores de las redes neuronales.

\subsection{Justificación}

En el trabajo de Kocevar et al.\cite{Kocevar2016} utilizaron las mediciones realizadas en las redes neuronales para determinar el fenotipo de EM del paciente estudiado obteniendo una sensibilidad del 76.499\% IC (65.9\% - 92.0\%).

% Esta frase no tiene sentido
El análisis de topología de la red neuronal mediante el uso de la teoria de grafos para caracterizar la red neuronal, puede suponer una manera nueva y alternativa de realizar el fenotipado de los pacientes con EM. Tras los resultados de Kocevar et al.\cite{Kocevar2016} y Muthuraman et al. \cite{Muthuraman2016}, es necesario mejorar la fiabilidad de estas herramientas y realizar estudios que confirmen y mejoren estos primeros resultados.


%------------------------------------------------
\section{Objetivo del estudio}

El objetivo principal de este estudio es comparar la especificidad y la sensibilidad de dos métodos de clasificación diagnóstica de pacientes con EM; el método propuesto por Kocevar et al.\cite{Kocevar2016} y una variante del mismo. El método de Kocevar et al.\cite{Kocevar2016} clasifica individuos en los distintos subtipos de EM o paciente sano, basándose en el análisis de la conectividad cerebral obtenida a partir del procesamiento de IRMN cerebrales. Este análisis de la conectividad, hace uso de grafos con nodos generados de manera aleatoria.

El modo en el que se estudia la conectividad de una red está supeditado a la topología de la misma \cite{Fornito, Zalesky2010}. Por ello, este trabajo propone hacer uso de particiones cerebrales ya conocidas para la generación de los nodos del grafo de forma sistemática. La variante propuesta hará uso de las particiones anatómico-funcionales del cerebro descritas por Diez et al.\cite{Diez2015}.

Las particiones anatómico-funcionales son la respuesta a los antiguos modelos de particiones que no responden a una estructuración del cerebro en base a sus funciones. Gracias a la IRMN y a la medición del flujo arterial cerebral que esta permite hacer en las secuencias BOLD \cite{Ogawa1990}, se han podido desarrollar modelos de particiones basándose en la funcionalidad del cerebro \cite{Heller2006}.

En 2015, Diez et al \cite{Diez2015} propusieron un modelo de particiones cerebrales teniendo en cuenta tanto las conexiones físicas entre las áreas como las áreas funcionales, creando un modelo del cerebro con 22 particiones estructuro-funcionales previamente no descritas. 


%------------------------------------------------
\section{Hipótesis}

Las hipótesis que se plantean en este estudio son la mejora de la sensibilidad y especificidad del método diagnóstico haciendo uso de particiones calculadas en base a la estructura y funcionalidad del cerebro para crear los nodos que determinarán la topología de la red. 
\vspace{1em} \\
Sensibilidad:
\begin{center}
$H_0: Sensibilidad_1 = Sensibilidad_2$ \\
$vs$ \\
$H_1: Sensibilidad_1 \neq Sensibilidad_2$
\end{center}
Especificidad:
\begin{center}
$H_0: Especificidad_1 = Especificidad_2$ \\
$vs$ \\
$H_1: Especificidad_1 \neq Especificidad_2$
\end{center}

%------------------------------------------------
\section{Material y Métodos}

\subsection{Diseño del estudio}

Esta propuesta constituye un estudio observacional, analítico, transversal que compara dos métodos de clasificación de pacientes con EM o paciente sano. Se hará uso de datos secundarios obtenidos de la base de datos desarrollada para el estudio de Kocevar et al. \cite{Kocevar2016} en 2016.

\subsection{Sujetos}
La población diana serán aquellas personas han desarrollado alguna sintomatología asociada a la EM y que se desee completar el estudio mediante otras herramientas diagnósticas. La muestra será un grupo de casos y controles que fueron reclutados en el Hospital Neurológico de Lyon entre 2014 y 2015 y que fueron diagnosticados de EM. Los controles reclutados fueron sujetos sanos con edad y sexo similares a los casos.

La muestra final la completaron 67 pacientes con EM (24 RR, 24 PP, 17 SP y 12 CIS) (29 hombres, 48 mujeres; edad media de 38,3 años, rango 21,5-48,7). Ademas se incluyeron 26 sujetos sanos a modo de control (HC), con edad y sexo similar a la de los pacientes con EM (9 hombres, 15 mujeres, con una edad media de 35,7 años, rango 21,6-56,5).

\begin{table}[hbt]
\centering
\begin{tabular}{lllll}
\toprule
\multicolumn{5}{l}{}\\
       & $n$ & $H$ & $M$  & $\bar{x} ~Edad$ \\
$HC$   & 24  & 15  & 09   & 35.7   \\
$CIS$  & 12  & 07  & 05   & 33.5   \\
$RR$   & 24  & 20  & 04   & 35.1   \\
$SP$   & 24  & 10  & 14   & 42.3   \\
$PP$   & 17  & 11  & 06   & 40.9   \\
\bottomrule
\end{tabular}
\vskip1em
\caption{
Datos demográficos\\
$n$: Numero de sujetos\\
$H$: Hombres\\
$M$: Mujeres\\
$Edad$: Edad media de diagnóstico\\
}
\label{tab:demographics}
\end{table}

\subsubsection{Criterios de inclusión}
Pacientes que hayan sido diagnosticados de EM y que cumplan las siguientes condiciones:
\begin{enumerate}[noitemsep]
\item Hace menos de 5 años que fueron diagnosticados de EM en el momento de la inclusión en el estudio
\item Haya sido diagnosticados haciendo uso de los criterios revisados de McDonald
\item Al menos una adquisición por IRMN de $1.5T$ en secuencias: $T_1$ 3D sagital y $DTI$ 2D Spin echo axial.
\item Dispuestos y capaces de dar su autorización por escrito.
\item Sean mayores de edad
\end{enumerate}

\subsubsection{Criterios de exclusión}
Se excluirá a los participantes que:
\begin{enumerate}[noitemsep]
\item Están embarazadas o planean quedarse embarazadas en un periodo menor a 6 meses
\item Hayan sido previamente diagnosticados de enfermedades neuropsiquiatricas, temporales o no incluyendo la depresión
\item Su presión arterial sistólica $<$ \textit{140 mmHg} o diastólica $<$ \textit{90 ~mmHg}, están recibiendo tratamiento o tienen antecedentes de hipertensión
\item Episodios previos de accidentes cerebrovasculares o vasculopatías
\item Esten tratados con antidepresivos o neurolépticos por otras patologías
\end{enumerate}


\subsection{Instrumentos de recolección de datos}

Los datos que se han de obtener para poder comparar ambos tests son por un lado las medidas que describen el grafo de la red neuronal y por el otro lado el diagnóstico de EM establecido. El primer paso es capturar las IRMN de los sujetos, generar un conectoma, y analizarlo como un grafo, obteniendo del grafo los valores de medición que lo describen matemáticamente. Por otra parte los pacientes obtendrán un diagnóstico clínico que será el que sirva de referencia para realizar la clasificación.

\begin{figure}[h]
	\centering
	\includegraphics[width=\linewidth]{img/procesado}
	\vspace{5mm} 
	\caption{Procesado de los datos para obtener el grafo}
	\label{fig:pipeline}
\end{figure}


Para la adquisición de IRMN se utilizó el modelo Siemens Sonata de $1,5T$ utilizando una bobina de cabeza de $8$ canales. El protocolo IRMN de adquisición consistió en una secuencia sagital 3D $T_1$ (1 x 1 x 1mm3, TE / TR = 4/2000ms) y una secuencia axial DTI de spin-echo 2D (TE / TR = 86 / 6900ms, 2 x 24 direcciones de difusión en gradiente, b = 1000, resolución espacial de 2,5 x 2,5 x 2,5 mm3) orientada en el plano AC/PC.

Tras la adquisición, el procesamiento de la imágenes se realizará con el software FSL \cite{Jenkinson2012} y la tractografía y obtención del grafo se realizará con el software MRtrix \cite{Tournier2012}. Esta última operación se repetirá en dos ocasiones, una con los nodos aleatorios de Kocevar et al. y otra con las pariticiones estructuro-funcionales de Diez et al. En la \{fig.\ref{fig:pipeline}\} queda reflejado este proceso.

Se pueden estimar varias métricas para medir las propiedades de grafo generado, para ello nos basaremos en aquellas propuestas por Rubinov et al.\cite{Rubinov2010}. Esta última operación se realizará con el paquete de análisis NetworkX \cite{Daducci2012} para obtener las medidas de segregación, integración, centralidad, resistencia y motif.


Para el diagnóstico y el fenotipado de los pacientes, nos basaremos en los datos recolectados por los clínicos. Estos utilizaron los criterios revisados de McDonald \cite{Mcdonald2001, Polman2011}, criterios basados tanto en la clínica de los pacientes así como en la observación de lesiones por un radiólogo en secuencia $T_2$ en la IRMN \{fig.\ref{fig:mcdonald}\}.  Para evaluar la discapacidad derivada de las lesiones de la EM, se utilizó la escala de EDSS \cite{Kurtzke1983}. El EDSS de un sistema de puntuación que valora diversos aspecto del paciente y asigna al paciente un resultado dentro de una escala semicuantitativa discreta \{fig.\ref{fig:edss}\}.

\begin{figure}[b]
	\centering
	\includegraphics[width=\linewidth]{img/mcdonald}
	\vspace{5mm} 
	\caption{Algoritmo (simplificado) de los criterios de McDonald}
	\label{fig:mcdonald}
\end{figure}

\begin{figure}[b]
	\centering
	\includegraphics[width=\linewidth]{img/edmus_edss}
	\vspace{5mm} 
	\caption{Ejemplo de escala EDSS}
	\label{fig:edss}
\end{figure}

\subsection{Variables finales}

Las variables de finales para cada paciente son el diagnóstico de EM y las 5 mediciones que describen el grafo, formando un total de 6 variables. El diagnóstico se trata de una variable cualitativa con 5 posibles casos, paciente sano (HC), clínicamente aislada (CIS), remitente-recurrente (RR), progresiva primaria (PP) y secundariamente progresiva (SP). Las 5 mediciones que describen el grafo se tratan de 5 variables cuantitativas continuas.

\subsection{Análisis estadístico}
Tras obtener las métricas que describen la red, estas pasarán junto con el diagnóstico clínico por un algoritmo de aprendizaje automatizado. Este algoritmo aprenderá a clasificar a los sujetos en una de las categorías de la esclerosis múltiple o paciente sano en base a estas métricas. En este caso el algoritmo utilizado es la máquina de vectores de soporte (SVM).

Con los resultados obtenidos a partir de este algoritmo y con la clasificación previa realizada por los clínicos, crearemos una matriz de confusión \ref{table:matriz} a partir de la cual se podrá calcular la sensibilidad y especificidad de estos tests. Repetiremos esta operación en dos ocasiones, una con la particiones aleatorias de Kocevar et al. y otra con las particiones de Diez et al. De esta manera obtendremos, la sensibilidad y especificidad para las dos herramientas de clasificación, pudiendo así compararlas entre sí. 

\begin{table}[h]
\centering
\label{matriz}
\begin{tabular}{rlllll}
\multicolumn{1}{l}{\textbf{}}                       & \multicolumn{5}{l}{\textbf{Valor predecido:}}            \\
\multicolumn{1}{l}{\textbf{Valor
Actual:}} & HC         & CIS        & PP         & SP         & RR         \\
HC                                                  & \textit{a} & \textit{b} & \textit{c} & \textit{d} & \textit{e} \\
CIS                                                 & \textit{f} & \textit{g} & \textit{h} & \textit{i} & \textit{j} \\
PP                                                  & \textit{k} & \textit{l} & \textit{m} & \textit{n} & \textit{o} \\
SP                                                  & \textit{p} & \textit{q} & \textit{r} & \textit{s} & \textit{t} \\
RR                                                  & \textit{u} & \textit{v} & \textit{w} & \textit{x} & \textit{y}
\end{tabular}
\vskip2em
\caption{Matriz de confusión entre el $test_{a/b}$ y el diagnóstico clínico}
\end{table}

En base a la especificidad y sensibilidad obtenida para los dos test de clasificación, pondremos a prueba las hipótesis del trabajo. Ambos tests se realizarán sobre los mismo pacientes, por lo que al tratarse de datos pareados estaría indicado realizar la prueba de McNemar como test estdístico. El criterio de significación utilizado será de $\alpha = 0,05$.


\subsection{Cronograma}

La duración aproximada para la realización de este estudio es de 11 meses. Debido a la complejidad de los datos a estudio es importante determinar el máximo numero de parámetros, para minimizar el efecto la variabilidad debida al azar y la correcta preparación de las herramientas con las que se llevarán a cabo el análisis. En el siguiente cronograma, \{fig.\ref{fig:gantt}\} en la página  \pageref{fig:gantt} se resumen las actividades a realizar durante este periodo de tiempo.

\begin{figure}[t]
	\centering
	\includegraphics[width=\linewidth]{img/gantt} 
	\caption{Cronograma de la investigación}
	\label{fig:gantt}
\end{figure}


%------------------------------------------------
\section{Discusión}

La combinación de la teoría de grafos y la naturaleza predictiva del aprendizaje automatizado está abriendo nuevas posibilidades para el diagnóstico y pronóstico de distintos trastornos neurológicos. Las nuevas capacidades técnicas de explorar la estimación de las redes neuronales están suponiendo un avance en la búsqueda de nuevos marcadores no invasivos \cite{Tymofiyeva2017,Fornito2015,Sun2016}.

Los primeros resultados satisfactorios en la aplicación de estos avances en en la EM \cite{Kocevar2016,Muthuraman2016} y las posibles mejoras que se pueden realizar sobre estas primeras herramientas parecen abrir la puerta al estudio de la conectómica humana y su aplicación clinica.

\subsection{Limitaciones}

Desde el punto de vista metodológico, el estudio presenta diversas limitaciones. La topología de la red en base a las pariticiones de Diez et al puede no mejorar los resultados. La dependencia de las redes a la forma en la que estas se han generado, es actualmente un debate abierto \cite{Zalesky2010}. 

Varias limitaciones del estudio son debidas al uso de la resonancia magnética nuclear. El bajo tesla del escáner (1.5$T$) puede contribuir a la variabilidad de las mediciones. Las IRMN tiene cierta variabilidad intrínseca con el mismo paciente, en la misma fecha, es por ello que se aconseja el uso del multiples capturas y trabajar con la media de ellas \cite{Landman2007}.

\subsection{Aspectos éticos}

Los principios éticos aceptados en investigación biomédica con seres humanos regirán este estudio. La fuente de la cual se obtendrán los datos fueron en el momento de su adquisición aprobados por el comité de ética local (CPP Sud-Est IV) y la agencia nacional francesa de medicina y productos sanitarios (ANSM). Se obtuvo el consentimiento informado por escrito de todos los sujetos.

La confidencialidad de los participantes y de sus datos será respetada en todo momento, y por ello se trabajará con datos anonimizados desde la fuente. Debido a la sensibilidad de los datos con los que se trabaja y las posibles futuras connotaciones que pudieran tener, se trabajará en obtener nuevamente un consentimiento informado de los paciente permitiendo la reutilización de sus datos con fines científicos.


%------------------------------------------------
\phantomsection
\section*{Agradecimientos}
\addcontentsline{toc}{section}{Agradecimientos}
Gracias a Amaia Lasa por su infinita paciencia y ayuda a simplificar ciertos aspectos del trabajo. A Eloi Martinez y Sara Llufriu, por ayudarme a trabajar en el IDIBAPS. Finalmente gracias Álvaro Abella y Javier de Oca, por creer en esta aventura y uniros a ella.
\endgroup


%----------------------------------------------------------------------------------------
%	REFERENCE LIST
%----------------------------------------------------------------------------------------
\phantomsection

\bibliographystyle{unsrt}
\bibliography{references}

%----------------------------------------------------------------------------------------

\end{document}